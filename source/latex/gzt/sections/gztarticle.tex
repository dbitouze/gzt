\chapter{Documentation de la classe destinée aux auteurs}
\label{cha:documentation-auteurs}

La \gls{smf} fournit la \Cls{gztarticle} destinée aux auteurs souhaitant publier
un article dans la revue \gzt*{}. Cette classe a pour but :
\begin{enumerate}
\item de reproduire fidèlement la maquette de la \gzt{}, permettant ainsi
  aux auteurs de pouvoir travailler la mise en page de leur document dans des
  conditions réelles ;
\item de fournir un certain nombre d'outils (commandes et environnements)
  destinés à faciliter la rédaction de documents, notamment ceux contenant des
  formules de mathématiques.
\end{enumerate}

\section{Article standard}
\label{sec:article-standard}

Nous commençons par décrire les éléments caractéristiques d'un article standard.

\subsection{Préparation du titre de l'article}
\label{sec:prep-de-lart}

Cette section liste les commandes, options et environnement permettant de
\emph{préparer} le titre de l'article ainsi que son éventuelle partie finale.

\subsubsection{Titre, sous-titre}
\label{sec:titre}

\begin{docCommand}[doc description=\string!]{title}{\oarg{options}}
  Cette commande définit le \meta{titre} de l'article. Celui-ci apparaît alors
  en début d'article et aussi comme métadonnée \enquote{Titre} du fichier
  \acrshort{pdf} correspondant.
\end{docCommand}

\begin{docCommand}{subtitle}{\oarg{options}}
  Cette commande définit l'éventuel \meta{sous-titre} de l'article. Celui-ci
  est automatiquement ajouté au titre.
\end{docCommand}

Les commandes \refCom{title} et \refCom{subtitle} admettent un argument
optionnel permettant de spécifier un (sous-)titre court au moyen de la clé
\refKey{short}.

\begin{docKey}{short}{=\meta{(sous-)titre court}}{pas de valeur par défaut,
    initialement vide}
  Cette clé définit un (sous-)titre \enquote{court} de l'article qui figure
  alors à la place du (sous-)titre \enquote{normal} dans le sommaire et en titre
  courant\footnote{En entête.}.
\end{docKey}

\begin{bodycode}
\title[short=Les travaux de Martin \textsc{Hairer}]{Martin \textsc{Hairer}, l'équation de KPZ et les structures de régularité}
\end{bodycode}

\subsubsection{Auteur(s)}
\label{sec:auteurs}

Un auteur d'article est spécifié au moyen de la commande \refCom{author}
suivante. En cas d'auteurs multiples, il  suffit de recourir à plusieurs
occurrences de cette commande.

\begin{docCommand}{author}{\oarg{options}\marg{auteur}}
  Cette commande, facultative, définit un \meta{auteur} de l'article. La saisie
  des prénom et nom de l'auteur doit se faire selon la syntaxe obligatoire
  suivante, identique à celle de \hologo{BibTeX} et \pkg{biblatex} :
  %
\begin{bodycode}[listing options={showspaces}]
~\meta{auteur}~=~\meta{Nom}~, ~\meta{Prénom}~
\end{bodycode}
\end{docCommand}
%
Les prénom et nom du ou des auteurs figurent en début d'article, les prénoms
étant abrégés au moyen de leurs initiales.

La commande \refCom{author} admet un argument optionnel permettant, pour chaque
auteur, de spécifier des renseignements complémentaires figurant en fin
d'article : son affiliation, sa photo, son email, sa page Web, sa biographie
express au moyen des clés \refKey{affiliation}, \refKey{photo}, \refKey{email},
\refKey{webpage} et \refKey{minibio} à séparer par des virgules :
\begin{bodycode}
\author[%
  affiliation=~\marg{affiliation(s)}~,%
  photo=~\meta{photo}~,%
  email=~\meta{email}~,%
  webpage=~\meta{page Web}~,%
  minibio=~\marg{biographie express}~%
]{~\meta{Nom}~, ~\meta{Prénom}~}
\end{bodycode}

\begin{docKey}{affiliation}{={\marg{affiliation(s)}}}{pas de valeur par défaut,
    initialement vide}
  Cette clé permet de spécifier une ou plusieurs affiliations. En cas
  d'affiliations multiples, celles-ci peuvent être séparées par la commande
  "\newline". La paire d'accolades n'est obligatoire qu'en cas (probable) de
  présence de virgule(s) dans la ou les \meta{affiliation(s)}.
\end{docKey}

\begin{docKey}{photo}{={\meta{photo}}}{pas de valeur par défaut,
    initialement vide}
  Cette clé permet de spécifier la photographie de l'auteur au moyen d'un
  fichier image. Les formats d'images acceptés sont ceux de \hologo{pdfLaTeX},
  à savoir \file{.jpg}, \file{.png} et \file{.pdf}.
\end{docKey}

\begin{docKey}{email}{={\meta{email}}}{pas de valeur par défaut,
    initialement vide}
  Cette clé permet de spécifier l'adresse courriel de l'auteur. Celle-ci doit
  être saisie telle quelle, sans recours aucun aux commandes "\url", "\href" ou
  assimilées du \Pkg{hyperref}.
\end{docKey}

\begin{docKey}{webpage}{={\meta{page Web}}}{pas de valeur par défaut,
    initialement vide}
  Cette clé permet de spécifier la page Web de l'auteur. Celle-ci doit être
  saisie telle quelle, sans recours aucun aux commandes "\url", "\href" ou
  assimilées du \Pkg{hyperref}.
\end{docKey}

\begin{docKey}{minibio}{={\marg{biographie express}}}{pas de valeur par défaut,
    initialement vide}
  Cette clé permet de spécifier la biographie express de l'auteur. La paire
  d'accolades n'est obligatoire qu'en cas (probable) de présence de virgule(s)
  dans la \meta{biographie express}.
\end{docKey}

\subsubsection{Remerciements}
\label{sec:remerciements}

\begin{docCommand}{acknowledgements}{\marg{remerciements}}
  Cette commande, facultative, permet de spécifier des remerciements pour un article.
\end{docCommand}

\subsubsection{Résumé}
\label{sec:resume}

\begin{docEnvironment}[doclang/environment content=résumé]{abstract}{}
  Cet environnement, facultatif, est destiné à recevoir le résumé de l'article.
\end{docEnvironment}

\subsection{Production du titre de l'article}
\label{sec:creation-du-titre}

Le titre proprement dit de l'article, regroupant tous les éléments saisis à la
\vref{sec:prep-de-lart}, est produit par la commande standard
\refCom{maketitle}.

\begin{docCommand}[doc description=\string!]{maketitle}{}
  Cette commande \emph{produit} le titre de l'article.
\end{docCommand}


\iffalse
%%% Local Variables:
%%% mode: latex
%%% eval: (latex-mode)
%%% ispell-local-dictionary: "fr_FR"
%%% TeX-master: "../gzt.dtx"
%%% End:
\fi
