\chapter{Documentation de la classe destinée aux auteurs}
\label{cha:documentation-auteurs}

La \gls{smf} fournit la \gztauthorcl{} destinée aux auteurs souhaitant publier
un article dans la
\href{http://smf4.emath.fr/Publications/Gazette/}{\gzt*{}}. Cette classe a pour
but :
\begin{enumerate}
\item de reproduire fidèlement la maquette de la \gzt{}, permettant ainsi
  aux auteurs de pouvoir travailler la mise en page de leur document dans des
  conditions réelles ;
\item de fournir un certain nombre d'outils (commandes et environnements)
  destinés à faciliter la rédaction de documents, notamment ceux contenant des
  formules de mathématiques.
\end{enumerate}

\section{Article standard}
\label{sec-article-standard}

Nous commençons par décrire les éléments caractéristiques d'un article standard.

\subsection{Préparation du \enquote{titre} de l'article}
\label{sec-prep-de-lart}

Cette section liste les commandes, options et environnement permettant de
\emph{préparer} le \enquote{titre} de l'article ainsi que son éventuelle partie
finale.

\subsubsection{Titre, sous-titre}
\label{sec-titre}

\begin{docCommand}[doc description=\mandatory]{title}{\oarg{options}}
  Cette commande définit le \meta{titre} de l'article. Celui-ci apparaît alors
  en début d'article et aussi comme métadonnée \enquote{Titre} du fichier
  \acrshort{pdf} correspondant.
\end{docCommand}

\begin{docCommand}{subtitle}{\oarg{options}}
  Cette commande définit l'éventuel \meta{sous-titre} de l'article. Celui-ci
  est automatiquement ajouté au titre.
\end{docCommand}

Les commandes \refCom{title} et \refCom{subtitle} admettent un argument
optionnel permettant de spécifier un (sous-)titre court au moyen de la clé
\refKey{short}.

\begin{docKey}{short}{=\meta{(sous-)titre court}}{pas de valeur par défaut,
    initialement vide}
  Cette clé définit un (sous-)titre \enquote{court} de l'article qui figure
  alors à la place du (sous-)titre \enquote{normal} dans le sommaire et en titre
  courant\footnote{En entête.}.
\end{docKey}

\begin{bodycode}
\title[short=Les travaux de Martin \textsc{Hairer}]{Martin \textsc{Hairer},
  l'équation de KPZ et les structures de régularité}
\end{bodycode}

\begin{dbremark}{Affichage des titre et sous-titre}{}
  Pour que les titre et sous-titre soient affichés, il est nécessaire de
  recourir à la commande habituelle \refCom{maketitle}.
\end{dbremark}

\subsubsection{Auteur(s)}
\label{sec-auteurs}

Un auteur d'article est spécifié au moyen de la commande \refCom{author}
suivante. En cas d'auteurs multiples, il  suffit de recourir à plusieurs
occurrences de cette commande.

\begin{docCommand}{author}{\oarg{options}\brackets{\meta{Nom}, \meta{Prénom}}}
  Cette commande, facultative, définit un auteur d'article.
\end{docCommand}
%
\begin{dbwarning}{Format des prénom et nom de l'auteur}{}
  On veillera à ce que :
  \begin{enumerate}
  \item la saisie des prénom et nom de l'auteur soit conforme à la syntaxe
    (identique à celle de \hologo{BibTeX} et \package{biblatex}) :
    %
\begin{bodycode}[listing options={showspaces}]
+\meta{Nom}+, +\meta{Prénom}+
\end{bodycode}
    %
  \item les éventuels accents figurent dans les \meta{Prénom} et
    \meta{Nom} ;
  \item le \meta{Nom} \emph{ne} soit \emph{pas} saisi en capitales
    (sauf pour la ou les majuscules) car il sera automatiquement
    composé en petites capitales.
  \end{enumerate}
\end{dbwarning}

\begin{dbremark}{Affichage des prénoms et noms du ou des auteurs}{}
  Pour que les prénoms et noms du ou des auteurs soient affichés, il est
  nécessaire de recourir à la commande habituelle \refCom{maketitle}.
\end{dbremark}

\subsubsection{Auteur(s) : détails}
\label{sec-auteurs-details}

La commande \refCom{author} admet un argument optionnel permettant, pour chaque
auteur, de spécifier un certain nombre de détails complémentaires : son affiliation, sa
photo, son email, sa page Web, sa biographie express au moyen des clés
respectives %^^A (à séparer par des virgules)
\refKey{affiliation}, \refKey{photo}, \refKey{email}, \refKey{webpage} et
\refKey{minibio} :
\begin{bodycode}
\author[%
  affiliation=+\marg{affiliation(s)}+,%
  photo=+\meta{photo}+,%
  email=+\meta{email}+,%
  webpage=+\meta{page Web}+,%
  minibio=+\marg{biographie express}+%
]{+\meta{Nom}+, +\meta{Prénom}+}
\end{bodycode}

\begin{docKey}{affiliation}{={\marg{affiliation(s)}}}{pas de valeur par défaut,
    initialement vide}
  Cette clé permet de spécifier une ou plusieurs affiliations. En cas
  d'affiliations multiples, celles-ci peuvent être séparées par la commande
  \docAuxCommand{newline}.
\end{docKey}

\begin{docKey}{photo}{={\meta{photo}}}{pas de valeur par défaut,
    initialement vide}
  Cette clé permet de spécifier la photographie de l'auteur au moyen d'un
  fichier image\footnote{Si ce fichier ne figure pas dans le dossier courant, il
    faut faire figurer le chemin (relatif) y menant.}. % Les formats d'images
  % acceptés sont ceux de \hologo{pdfLaTeX}, à savoir \file{.jpg}, \file{.png} et
  % \file{.pdf}.
\end{docKey}

\begin{docKey}{email}{={\meta{email}}}{pas de valeur par défaut,
    initialement vide}
  Cette clé permet de spécifier l'adresse courriel de l'auteur.
\end{docKey}

\begin{docKey}{webpage}{={\meta{page Web}}}{pas de valeur par défaut,
    initialement vide}
  Cette clé permet de spécifier la page Web de l'auteur.
\end{docKey}

\begin{docKey}{minibio}{={\marg{biographie express}}}{pas de valeur par défaut,
    initialement vide}
  Cette clé permet de spécifier la biographie express de l'auteur.
\end{docKey}

\begin{dbwarning}{Paires d'accolades
    % des clés \refKey{affiliation} et \refKey{minibio}
    (relativement) obligatoires}{}
  Au cas (probable) où les valeurs \meta{affiliation(s)} et \meta{biographie
    express} des clés \refKey{affiliation} et \refKey{minibio} contiennent des
  virgules, les paires d'accolades les entourant sont obligatoires.
\end{dbwarning}

\begin{dbwarning}{Courriels et pages Web à saisir tels quels}{}
  Les valeurs \meta{email} et \meta{webpage} des clés \refKey{email} et
  \refKey{webpage} doivent être saisies telles quelles, sans recours aucun aux
  commandes \docAuxCommand{url}, \docAuxCommand{href} ou assimilées des packages
  \package{url} ou \package{hyperref}.
\end{dbwarning}

\begin{dbremark}{Affichage des détails complémentaires sur les auteurs}{}
  Pour que les détails complémentaires précédents soient affichés, il est
  nécessaire de recourir aux commandes \refCom{printauthorsdetails} ou
  \refCom{printbibliography}, destinées à être utilisées en fin d'article.

  Ces commandes affichent également d'éventuels remerciements à spécifier au
  moyen de la commande \refCom{acknowledgements} suivante.
\end{dbremark}

\begin{docCommand}{acknowledgements}{\marg{remerciements}}
  Cette commande, facultative, permet de spécifier des \meta{remerciements} pour
  un article.
\end{docCommand}

\begin{bodycode}
\acknowledgements{%
  L'auteur remercie Frédéric Patras, pour les nombreuses discussions qu'il
  a eues avec lui sur le sujet. Il remercie également le relecteur anonyme,
  qui a lu le texte avec un très grand soin, et dont les commentaires et
  suggestions ont été très utiles.%
}
\end{bodycode}

\subsubsection{Résumé}
\label{sec-resume}

\begin{docEnvironment}[doclang/environment content=résumé]{abstract}{}
  Cet environnement, facultatif, est destiné à recevoir le résumé de l'article.
\end{docEnvironment}

\subsection{Production du titre de l'article}
\label{sec-creation-du-titre}

Le titre proprement dit de l'article, regroupant tous les éléments saisis à la
\vref{sec-prep-de-lart}, est produit par la commande standard
\refCom{maketitle}.

\begin{docCommand}[doc description=\mandatory]{maketitle}{}
  Cette commande \emph{produit} l'affichage du \enquote{titre} de l'article,
  c'est-à-dire :
  \begin{itemize}
  \item son titre et son éventuel sous-titre (commandes \refCom{title} et
    \refCom{subtitle}) ;
  \item son ou ses éventuels auteurs, sous la forme de leurs noms et
    prénoms\footnote{Les prénoms sont alors abrégés au moyen de leurs
      initiales.} (commande(s) \refCom{author}) ;
  \item son éventuel résumé (environnement \refEnv{abstract}).
  \end{itemize}
\end{docCommand}

L'exemple suivant illustre la plupart des commandes et options vues jusqu'ici.

\begin{bodycode}
\title[short=Les travaux de Manjul \textsc{Bhargava}]{Manjul \textsc{Bhargava},
  anneaux de petit rang et courbes elliptiques}
%
\author[%
affiliation={%
  Univ. Bordeaux, IMB, UMR 5251, F-33400 Talence, France\newline CNRS, IMB, UMR
  5251, F-33400 Talence, France\newline INRIA, F-33400 Talence, France%
},%
photo=Belabas,%
email=Karim.Belabas@math.u-bordeaux.fr,%
minibio={%
  Karim Belabas est professeur à l'université de Bordeaux.  Ses centres
  d'intérêts sont la théorie des nombres sous toutes ses formes et le calcul
  formel.  Il développe le système libre PARI/GP.%
}%
]{Belabas, Karim}
%
\author[%
affiliation={%
  Laboratoire de Mathématiques de Besançon, Facultés des sciences et techniques,
  CNRS, UMR 6623, 16 route de Gray, 25030 Besançon, France%
},%
photo=Delaunay,%
email=Christophe.Delaunay@univ-fcomte.fr,%
minibio={%
  Christophe Delaunay est professeur à l'université de Franche-Comté et membre
  du laboratoire de mathématiques de Besançon. Il est spécialiste de théorie des
  nombres.%
}%
]{Delaunay, Christophe}
%
\begin{abstract}
  Manjul Bhargava a reçu la médaille Fields au congrès international de Séoul
  \enquote{pour avoir développé de nouvelles méthodes en géométrie des nombres,
    qu'il a appliquées au comptage des anneaux de petit rang et pour borner le
    rang moyen de courbes elliptiques}. Cet article est un survol d'une partie
  de ses travaux.
\end{abstract}
%
\maketitle
\end{bodycode}

\subsection{Affichage des détails complémentaires sur les auteurs}

\begin{docCommand}{printauthorsdetails}{}
  Cette commande, facultative, \emph{produit} l'affichage :
  \begin{itemize}
  \item des détails complémentaires sur les auteurs (options
    \refKey{affiliation}, \refKey{photo}, \refKey{email}, \refKey{webpage} et
    \refKey{minibio}) ;
  \item des éventuels remerciements (commande \refCom{acknowledgements}) ;
  \end{itemize}
  tels que définis à la \vref{sec-auteurs-details}.
\end{docCommand}

\begin{dbwarning}{Commande \protect\docAuxCommand*{printauthorsdetails} inutile
    en cas de bibliographie}{}
  En cas de bibliographie (cf. commande \refCom{printbibliography}), l'usage de
  la commande \refCom{printauthorsdetails} est inutile car l'affichage des
  détails complémentaires sur les auteurs est alors automatique.
\end{dbwarning}

\section{Article sans auteur mais avec signature \enquote{académique}}
\label{sec:article-sans-auteur}

Les articles standard précédents sont écrits par une ou plusieurs personnes
physiques clairement identifiées en tant qu'auteurs.

D'autres types d'articles n'ont pas véritablement d'auteurs mais sont plutôt
\enquote{signés} par, par exemple, un comité. Pour ce type d'article, on ne
recourra pas à la commande \refCom{author}, mais plutôt à la commande
\refCom{academicsignature} suivante.

\begin{docCommand}{academicsignature}{\marg{signature}}
  Cette commande permet de spécifier une \meta{signature} académique.
\end{docCommand}

\begin{bodycode}
\academicsignature{%
  Au nom du comité éditorial, Serge Nicaise (éditeur en chef) et Nicolas
  Wicker (directeur technique).%
}
\end{bodycode}

\begin{dbwarning}{}{}
  La commande \refCom{author} \enquote{prépare} un auteur qui n'est affiché que
  là où est employée la commande \refCom{maketitle}. Au contraire, la commande
  \refCom{academicsignature} affiche \enquote{immédiatement} la \meta{signature}
  passée en argument.
\end{dbwarning}

\section{Entretiens}
\label{sec:entretiens}

Les entretiens sont des articles en général sans auteur mais les personnes ayant
recueilli les propos peuvent être signalées dans le résumé (cf. environnement
\refEnv{abstract}) :

\begin{bodycode}
\begin{abstract}
  Propos recueillis à Séoul, pendant l'ICM, par Boris \surname{Adamczewski} et
  Gaël \surname{Octavia}.
\end{abstract}
\end{bodycode}

Les questions

\begin{docCommand}{academicsignature}{\marg{signature}}
  Cette commande permet de spécifier une \meta{signature} académique.
\end{docCommand}

\begin{dbwarning}{}{}
  La commande \refCom{author} \enquote{prépare} un auteur qui n'est affiché que
  là où est employée la commande \refCom{maketitle}. Au contraire, la commande
  \refCom{academicsignature} affiche \enquote{immédiatement} la \meta{signature}
  passée en argument.
\end{dbwarning}

\iffalse
%%% Local Variables:
%%% mode: latex
%%% eval: (latex-mode)
%%% ispell-local-dictionary: "fr_FR"
%%% TeX-master: "../gzt.dtx"
%%% End:
\fi
