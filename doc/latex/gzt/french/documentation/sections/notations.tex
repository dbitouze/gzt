\etocsettocdepth.toc{section}%
%
\section{Notations, syntaxe, terminologie et codes couleurs}\label{sec-synt-term-notat}

Ce chapitre précise les notations, syntaxe, terminologie et codes couleurs de
la présente documentation.


\subsection{Commandes, environnements, clés, valeurs}\label{sec-comm-envir-cles}

Les commandes, environnements, clés et valeurs de clés sont systématiquement
composés en fonte à chasse fixe. En outre, pour plus facilement les
distinguer, ils figurent avec des couleurs propres :
\begin{itemize}
\item les commandes en bleu : \docAuxCommand*{commande} ;
\item les environnements en \enquote{sarcelle} :
  \docAuxEnvironment*{environnement} ;
\item les clés en pourpre : \docAuxKey*{clé} ;
\item les valeurs des clés en violet : \docValue*{valeur}.
\end{itemize}

\subsection{Arguments génériques}
\label{sec-arguments-generiques}

Pour expliquer le rôle d'une commande, il est parfois nécessaire d'indiquer
à quoi celle-ci s'applique, autrement dit quel en est l'argument générique.
Un tel argument est composé :
\begin{itemize}
\item en fonte à chasse fixe ;
\item en italique ;
\item entre chevrons simples ;
\end{itemize}
le tout en marron, ainsi : \meta{argument générique}.

\subsection{Liens hypertextes}
\label{sec-liens-hypertextes}

Les liens hypertextes figurent en couleur, ainsi :
\href{http://gte.univ-littoral.fr/members/dbitouze/pub/latex}{lien hypertexte}.
La plupart des références aux commandes, environnements et clés définis dans la
présente documentation, sont des liens hypertextes, surmontés du numéro de page
où se trouve la cible correspondante (sauf si elle se situe sur la même page) :
\begin{itemize}
\item \refCom{author} ;
\item \refEnv{abstract}.
\end{itemize}


\subsection{Éléments \enquote{obligatoires}}
\label{sec-comm-oblig}

L'icône \mandatory{}, figurant en regard de certains éléments (commandes ou
environnements), indique que ceux-ci sont \enquote{obligatoires}.

\subsection{Codes sources}
\label{sec-codes-sources}

Les exemples qui illustrent la présente documentation sont constitués de codes
sources et, le cas échéant, des \enquote{copies d'écran} correspondantes.
% Ceux-ci proviennent le plus souvent du spécimen de document composé avec la
% \yatcl, fourni avec l'ensemble de la classe
% (cf. \vref{sec-specimen-arborescence}).

Ces codes sources figurent dans des cadres de couleur bleu :
\begin{itemize}
\item non ombrés s'ils doivent être saisis dans le corps du document ;
\item ombrés s'ils doivent être saisis en préambule du fichier (maître).
\end{itemize}
Ces cadres pourront en outre comporter d'éventuels titres :
\begin{multicols}{2}
\begin{bodycode}
"\meta{code source}"
\end{bodycode}
\begin{bodycode}[title=\meta{titre}]
"\meta{code source}"
\end{bodycode}
\begin{preamblecode}
"\meta{code source à insérer en préambule}"
\end{preamblecode}
\begin{preamblecode}[title=\meta{titre}]
"\meta{code source à insérer en préambule}"
\end{preamblecode}
\end{multicols}

\subsection{Espaces dans les codes sources}
\label{sec-espaces-dans-les}

Pour éviter certaines confusions, les espaces dans les codes sources devant
être saisis au clavier sont parfois matérialisés au moyen de la marque
\lstinline[showspaces]+ +.

\subsection{Options}
\label{sec-options}

La présente classe, ainsi que certaines de ses commandes et certains de ses
environnements, peuvent être modulés au moyen d'options, ou listes d'options
(séparées par des virgules). Ces options se présentent sous la forme
\meta{clé}|=|\meta{valeur} et la \meta{valeur} passée à une \meta{clé} peut
être :
%^^A \begin{description}
%^^A \item[ne prennent pas de valeur.] Une telle option, par exemple nommée
%^^A   \refKey{option}, est alors documentée selon la syntaxe suivante:
%^^A     \begin{docKey*}{option}{}{\meta{valeurs par défaut et initiale}}
%^^A       \meta{Description de \refKey{option}}
%^^A     \end{docKey*}
%^^A \item[prennent des valeurs.] Une telle option se présente alors sous la forme
%^^A   \meta{clé}"="\meta{valeur}. Les valeurs passées à une clé peuvent être :
\begin{description}
\item[libre.] Si une telle \meta{clé} est (pour l'exemple) nommée
  \refKey{freekey}, elle est alors documentée selon la syntaxe suivante :
  \begin{docKey*}{freekey}{=\meta{valeur}}{\meta{valeurs par défaut et initiale}}
    \meta{Description de \refKey{freekey}}
  \end{docKey*}
\item[imposée] (parmi une liste de valeurs possibles). Si une telle \meta{clé} est
   (pour l'exemple) nommée \refKey{choicekey} et de valeurs imposées
  \docValue*{valeur1}, \docValue*{valeur2}, ..., \docValue*{valeurN}, elle est alors
  documentée selon la syntaxe suivante\footnote{Comme souvent en informatique,
    la barre verticale séparant les valeurs possibles signifie \enquote{ou}.} :
  \begin{docKey*}{choicekey}{=\docValue*{valeur1}\textbar\docValue*{valeur2}\textbar...\textbar\docValue*{valeurN}}{\meta{valeurs par défaut et initiale}}
    \meta{Description de \refKey{choicekey} et de ses valeurs possibles}
  \end{docKey*}
\end{description}
%^^A \end{description}

Les \meta{valeurs par défaut et initiale} d'une clé sont souvent précisées
(entre parenthèses en fin de ligne). Elles indiquent ce que la clé vaut :
\begin{description}
\item[par défaut] c'est-à-dire lorsque la clé \emph{est} employée, mais
  \emph{seule} c'est-à-dire sans qu'une valeur explicite lui soit passée ;
\item[initialement] c'est-à-dire lorsque la clé \emph{n'est pas} employée.
\end{description}

%%% Local Variables:
%%% mode: latex
%%% eval: (latex-mode)
%%% ispell-local-dictionary: "fr_FR"
%%% TeX-master: "../gzt-fr.tex"
%%% End:
