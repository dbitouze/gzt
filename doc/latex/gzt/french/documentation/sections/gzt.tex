\chapter{Documentation}
\label{cha:documentation}

\section{Configuration de la classe}

Pour configurer la classe, on dispose de la commande semi-globale
\refCom{issuesetup}, à utiliser autant de fois que souhaité, de préférence en
préambule mais également au fil du document. Elle permet d'apporter des
modifications :
\begin{itemize}
\item ordinaires, par exemple pour spécifier divers aspects de l'édition en
  cours (notamment son numéro) ;
\item extraordinaires, par exemple pour modifier une de ses expressions clés.
\end{itemize}

\begin{docCommand}{issuesetup}{\marg{configuration}}
  Cette commande permet de configurer la classe, la \meta{configuration} se
  faisant au moyen de clés/valeurs.
\end{docCommand}

\section{Configuration de l'édition en cours}

Pour configurer l'édition en cours, on dispose des clés \refKey{number}
(obligatoire) et \refKey{month} et \refKey{year} (facultatives), à passer à la
commande \refCom{issuesetup}.

\begin{docKey}{number}{=\meta{nombre}}{pas de valeur par défaut,
    initialement vide}
\end{docKey}

\begin{docKey}{month}{=\meta{nombre}}{pas de valeur par défaut,
    initialement vide}
\end{docKey}

\begin{docKey}{year}{=\meta{nombre}}{pas de valeur par défaut,
    initialement vide}
\end{docKey}

\section{Personnes impliquées dans le journal}
\label{sec:pers-impl-dans}

On liste ici les commandes permettant de spécifier les personnes impliquées dans
le journal.

Les commandes \refCom{editorinchief}, \refCom{editor} et \refCom{president}
suivantes permettent de spécifier respectivement un (ou plusieurs) rédacteur en
chef, un (ou plusieurs) rédacteur et le président de la \gls{smf} au moyen de
leurs :
\begin{itemize}
\item \meta{prénom} ;
\item \meta{nom} ;
\item \meta{institut} ;
\item \meta{courriel} ;
\end{itemize}
et, le cas échéant, de leur \meta{spécialité}.

\begin{docCommand}{editorinchief}{\oarg{specialité}\marg{prénom}\marg{nom}\marg{institut}\marg{courriel}}
  Cette commande permet de spécifier un rédacteur en chef.
\end{docCommand}

\begin{docCommand}{editor}{\oarg{specialité}\marg{prénom}\marg{nom}\marg{institut}\marg{courriel}}
  Cette commande permet de spécifier un rédacteur.
\end{docCommand}

\begin{docCommand}{president}{\oarg{specialité}\marg{prénom}\marg{nom}\marg{institut}\marg{courriel}}
  Cette commande permet de spécifier le président de la \gls{smf}.
\end{docCommand}

Les commandes \refCom{classdesigner}, \refCom{classmaintainer} et
\refCom{graphicdesigner} suivantes, permettant de spécifier respectivement le
concepteur de la classe \LaTeX, le ou les maintaineurs de la classe \LaTeX et le
concepteur de la maquette, sont similaires aux commandes précédentes, sauf
qu'aucune spécialité ne peut être spécifiée.

\begin{docCommand}{classdesigner}{\marg{prénom}\marg{nom}\marg{institut}\marg{courriel}}
  Cette commande permet de spécifier le concepteur de la classe \LaTeX.
\end{docCommand}

\begin{docCommand}{classmaintainer}{\marg{prénom}\marg{nom}\marg{institut}\marg{courriel}}
  Cette commande permet de spécifier le ou les mainteneurs de la classe \LaTeX.
\end{docCommand}

\begin{docCommand}{graphicdesigner}{\marg{prénom}\marg{nom}\marg{institut}\marg{courriel}}
  Cette commande permet de spécifier le concepteur de la maquette.
\end{docCommand}

%%% Local Variables:
%%% mode: latex
%%% eval: (latex-mode)
%%% ispell-local-dictionary: "fr_FR"
%%% TeX-master: "../gzt-fr.tex"
%%% End:
